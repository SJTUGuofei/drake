\documentclass{article}
\usepackage{amsmath, amsfonts, leftidx}
\begin{document}
Joint 1 connects base link (link 0) and link 1, by rotating about -y axis of the joint frame by angle $\theta_1$. The transform from link 1 to link 0 is
\begin{align}
\leftidx{^0}X^1 = \begin{bmatrix}c_1& 0 & -s_1 & 0\\
                                s_1& 0 & c_1 & 0\\
                                  0 & -1&  0  & l_0\\
                                  0 & 0 &  0  & 1\end{bmatrix}
\end{align}
where $c_1 = \cos\theta_1, s_1=\sin\theta_1$.

Joint 2 connects link 2 and link 1, by rotating about  z axis of the joint frame by angle $\theta_2$.  The transform from link 2 to link 1 is
\begin{align}
\leftidx{^1}X^2 = \begin{bmatrix}
	c_2 & -s_2 & 0  & l_{1x}\\
	s_2 & c_2  & 0  & -l_{1y}\\
	0   &  0   & 1  & 0\\
	0   & 0    & 0  & 1
		\end{bmatrix}
\end{align}
where $c_2 = \cos\theta_2, s_2 = \sin\theta_2$.

Joint 3 connects link 3 and link 2, by rotating about z axis of the joint frame by angle $\theta_3$. The transform from link 3 to link 2 is
\begin{align}
	\leftidx{^2}X^3 = \begin{bmatrix} 
		c_3 & -s_3 & 0 & 0\\
		s_3 &  c_3 & 0 & -l_2\\
		0 & 0 & 1 & 0\\
		0 & 0 & 0 & 1
	\end{bmatrix}
\end{align}
where $c_3 = \cos\theta_3, s_3 = \sin\theta_3$.

Joint 4 connects link 4 and link 3, by rotating about x axis of the joint frame by angle $\theta_4$. The transform from link 4 to link 3 is
\begin{align}
	\leftidx{^3}X^4 = \begin{bmatrix}
		1 & 0 & 0 & l_3\\
		0 & c_4 & -s_4 & 0\\
		0 & s_4 &  c_4 & 0\\
		0 & 0 & 0 & 1
	\end{bmatrix}
\end{align}

Joint 5 connects link 5 and link 4, by rotating about -z axis of the joint frame by angle $\theta_5$. The transform from link 5 to link 4 is
\begin{align}
	\leftidx{^4}X^5 = \begin{bmatrix}
		c_5 & s_5 & 0 & l_4\\
		-s_5 & c_5 & 0 & 0\\
		0 & 0 & 1 & 0\\
		0 & 0 & 0 & 1
	\end{bmatrix}
\end{align}
where $c_5=\cos\theta_5,s_5=\sin\theta_5$.

Joint 6 connects link 6 and link 5, by rotating about x axis of the joint frame by angle $\theta_6$. The transform from link 6 to link 5 is
\begin{align}
	\leftidx{^5}X^6 = \begin{bmatrix}
		1 & 0 & 0 & 0\\
		0 & c_6 & -s_6 & 0\\
		0 & s_6 & c_6 & 0\\
		0 & 0 & 0 & 1
	\end{bmatrix}
\end{align}
where $c_6=\cos\theta_6, s_6=\sin\theta_6$.


In end effector pose $\leftidx{^0}X^6$ can be written as
\begin{align}
	\leftidx{^0}p^6 = \begin{bmatrix}
		c_1  (l_{1x} + l_2 s_2 + l_3  c_{23} + l_4  c_{23})\\
		s_1  (l_{1x} + l_2 s_2 + l_3  c_{23} + l_4  c_{23})\\
		l_0 + l_{1y} + l_2 c_2 - l_3  s_{23} - l_4  s_{23}
	\end{bmatrix}
\end{align}
We first notice that $\leftidx{^0}X^6$ has the property
\begin{align}
	\tan\theta_1 = \arctan2(\leftidx{^0}p^6_y, \leftidx{^0}p^6_x)
\end{align}

If we define
\begin{align}
	a_0 = \leftidx{^0}p^6_z - l_0 - l_{1y}\\
	b_0 = \frac{\leftidx{^0}p^6_x}{c_1}-l_{1x}
\end{align}
We then get
\begin{align}
	a_0 = l_2c_2 - (l_3 + l_4)s_{23}\\
	b_0 = l_2s_2+(l_3 + l_4)c_{23}
\end{align}

whichi leads to
\begin{align}
	a_0c_2+b_0s_2=l_2-(l_3+l_4)s_3\\
	b_0c_2-a_0s_2 = (l_3+l_4)c_3
\end{align}
and
\begin{align}
	a_0c_2+b_0s_2 = \frac{a_0^2 + b_0^2 - l_2^2 - (l_3 + l_4)^2}{2l_2}
\end{align}
We can compute $\theta_2$ from this equation, and $\theta_3$ from equations above, once $\theta_2$ is known.

The orientation of link 6 satisfies
\begin{align}
	\leftidx{^0}R^6(1,1) = &c_1(c_{23}c_5 + s_{23}c_4s_5) + s_1s_4s_5\\
	\leftidx{^0}R^6(2,1) = &-c_1s_4s_5 + s_1(c_{23}c_5 + s_{23}c_4s_5)\\
	\leftidx{^0}R^6(3,1) = &c_{23}c_4s_5 - s_{23}c_5
\end{align}

We get
\begin{align}
	c_5 = c_1c_{23}\leftidx{^0}R^6(1,1) + s_1c_{23}\leftidx{^0}R^6(2,1)-s_{23}\leftidx{^0}R^6(3,1)
\end{align}

Then if $s_5$ is not zero, we can compute $\theta_4$ as
\begin{align}
	\theta_4 = atan2\left(\frac{s_1\leftidx{^0}R^6(1,1)-c_1\leftidx{^0}R^6(2,1)}{s_5},\frac{\leftidx{^0}R^6(3,1)+s_{23}c_5}{c_{23}s_5}\right)
\end{align}
\end{document}
